%
% Chapitre "recommandations pour les projets"
%

\newcounter {regle}
% \renewcommand {\theregle} {\arabic {regle}}

\newcommand {\regle} [1] {
    \refstepcounter {regle}
    {\bf Règle \theregle} : {\em #1} \par
}



Afin de vous aider dans la rédaction de vos projets, voici quelques
règles de bon sens que nous vous conseillons de suivre.  Celles-ci
résultent de la lecture de nombreux projets de vos camarades des années
antérieures.

%%%%%%%%%%%%%%%%%%%%%%%%%%%%%%%%%%%%%%%%%%%%%%%%%%%%%%%%%%%%%%%%%%%%%%%%%%%%%%

\regle {Ne dépassez pas 80 caractères par ligne}

Lorsqu'on imprime votre projet, les lignes trop longues sont simplement
tronquées. Si le correcteur ne voit qu'une partie de votre projet, il
ne vous mettra qu'une partie de la note...

%%%%%%%%%%%%%%%%%%%%%%%%%%%%%%%%%%%%%%%%%%%%%%%%%%%%%%%%%%%%%%%%%%%%%%%%%%%%%%

\regle {Indentez avec {\em au moins} 4 espaces}

La largeur d'indentation est un élément important de la lisibilité de
vos programmes. Une largeur de 4 espaces est généralement considérée
comme la largeur minimum d'indentation.

%%%%%%%%%%%%%%%%%%%%%%%%%%%%%%%%%%%%%%%%%%%%%%%%%%%%%%%%%%%%%%%%%%%%%%%%%%%%%%

\regle {Utilisez des espaces pour améliorer la lisibilité}

Par exemple, plutôt qu'écrire~: \par
\hspace* {20mm} \verb|fn(arg1,arg2);| \par
ajoutez quelques espaces en respectant les règles typographiques~: \par
\hspace* {20mm} \verb|fn (arg1, arg2) ;|

%%%%%%%%%%%%%%%%%%%%%%%%%%%%%%%%%%%%%%%%%%%%%%%%%%%%%%%%%%%%%%%%%%%%%%%%%%%%%%

%    - utiliser des noms de variable/fonction lisibles
%
%	Par exemple, ne pas nommer une variable :
%		VariableAvecUnNomPasLisibleDuTout
%	mais :
%		variable_avec_un_nom_tres_lisible

%%%%%%%%%%%%%%%%%%%%%%%%%%%%%%%%%%%%%%%%%%%%%%%%%%%%%%%%%%%%%%%%%%%%%%%%%%%%%%

\regle {\'Evitez de perdre trop de temps sur la forme des commentaires}

Les commentaires sont importants... mais pas leur forme. Chaque année,
on voit dans les projets des commentaires présentés avec beaucoup
de soin, comme par exemple des cadres~:

\begin {quote}
\begin {verbatim}
/************************************
 * fonction pour faire ceci ou cela *
 ************************************/
\end{verbatim}
\end {quote}

ou des alignements~:

\begin {quote}
\begin {verbatim}
/* ceci est un commentaire                     */
/* ceci est un deuxieme commentaire            */
\end{verbatim}
\end {quote}

Cela ne sert pas à grand-chose, si ce n'est à vous faire perdre du
temps, et à enlaidir votre programme si par malheur les alignements ne
sont pas parfaits. Alors, faites au plus simple sans fioriture.

%%%%%%%%%%%%%%%%%%%%%%%%%%%%%%%%%%%%%%%%%%%%%%%%%%%%%%%%%%%%%%%%%%%%%%%%%%%%%%

\regle {\'Eliminez les commentaires inutiles}

Vous n'êtes pas payé au kilo de commentaire. Mieux vaut souvent pas
de commentaire plutôt que des commentaires comme~:

\begin {quote}
\begin {verbatim}
int x ;     /* declaration */
x++ ;       /* incrementation de x */
\end{verbatim}
\end {quote}

N'oubliez pas que la personne qui relit connaît déjà le langage~C.

%%%%%%%%%%%%%%%%%%%%%%%%%%%%%%%%%%%%%%%%%%%%%%%%%%%%%%%%%%%%%%%%%%%%%%%%%%%%%%

\regle {\'Ecrivez des commentaires utiles}

Les commentaires doivent expliquer le fonctionnement de votre programme,
ou éventuellement quelques points épineux. Mettez vous à la place de la
personne qui relit votre programme.
Si on enlève les instructions, on doit pouvoir comprendre le fonctionnement
du programme à partir des commentaires.

%%%%%%%%%%%%%%%%%%%%%%%%%%%%%%%%%%%%%%%%%%%%%%%%%%%%%%%%%%%%%%%%%%%%%%%%%%%%%%

\regle {Présentez les fonctions dans un ordre logique}

Par exemple, choisissez de présenter les fonctions, dans chaque fichier
source, du plus haut niveau vers le plus bas niveau, et restez
homogènes. Votre lecteur doit pouvoir retrouver facilement vos
fonctions.


%%%%%%%%%%%%%%%%%%%%%%%%%%%%%%%%%%%%%%%%%%%%%%%%%%%%%%%%%%%%%%%%%%%%%%%%%%%%%%

\regle {Soignez l'orthographe et la grammaire}

Passer le correcteur orthographique sur votre rapport, c'est bien.
Le relire soigneusement, c'est mieux. Essayez de trouver les erreurs
avant la personne chargée de corriger votre projet~!


%%%%%%%%%%%%%%%%%%%%%%%%%%%%%%%%%%%%%%%%%%%%%%%%%%%%%%%%%%%%%%%%%%%%%%%%%%%%%%

\regle {L'orthographe, ce n'est pas seulement dans le rapport}

En particulier, soignez les commentaires...

%%%%%%%%%%%%%%%%%%%%%%%%%%%%%%%%%%%%%%%%%%%%%%%%%%%%%%%%%%%%%%%%%%%%%%%%%%%%%%

\regle {Simplifiez l'écriture}

Simplifiez votre style d'écriture : attention aux phrases ampoulées,
ambitieuses. Faites simple...

%%%%%%%%%%%%%%%%%%%%%%%%%%%%%%%%%%%%%%%%%%%%%%%%%%%%%%%%%%%%%%%%%%%%%%%%%%%%%%

\regle {Pas de franglais}

Ne faites pas de mélange entre le français et l'anglais, que ce soit
dans les noms de variables, de fonctions, dans les commentaires, etc.

Exemples à ne pas suivre : {\tt clean\_liste\_utilisateurs()}, {\tt
affiche\_users()}, etc.

%%%%%%%%%%%%%%%%%%%%%%%%%%%%%%%%%%%%%%%%%%%%%%%%%%%%%%%%%%%%%%%%%%%%%%%%%%%%%%

\regle {Faites des programmes robustes}

Un bon programme est un programme robuste. Il ne s'arrête jamais
inopinément avec un {\tt core}, tous les codes de retour des
primitives et des fonctions de librairie sont testés.

Un bon outil pour tester la robustesse est PureCoverage. Utilisez-le~!

%%%%%%%%%%%%%%%%%%%%%%%%%%%%%%%%%%%%%%%%%%%%%%%%%%%%%%%%%%%%%%%%%%%%%%%%%%%%%%

\regle {Attention au débordement de chaînes}

Une des erreurs les plus fréquentes est le débordement de chaînes de
caractères ou de tableaux (par exemple, l'introduction d'une ligne trop
longue par l'utilisateur, etc.).

Un bon outil pour tester le débordement de chaînes est Purify. Utilisez-le~!

%%%%%%%%%%%%%%%%%%%%%%%%%%%%%%%%%%%%%%%%%%%%%%%%%%%%%%%%%%%%%%%%%%%%%%%%%%%%%%

\regle {Pour la portabilité, rien ne vaut l'expérience}

Essayez donc de compiler votre programme avec le maximum de compilateurs
(au moins le compilateur fourni par le constructeur, et le compilateur
de GNU) pour être sûr que vous n'utilisez pas des caractéristiques de
tel ou tel compilateur.  Essayez de le compiler également sur le maximum
de machines ayant des systèmes différents.


%%%%%%%%%%%%%%%%%%%%%%%%%%%%%%%%%%%%%%%%%%%%%%%%%%%%%%%%%%%%%%%%%%%%%%%%%%%%%%

\regle {Attention aux "limites raisonables"}

Les limites qui peuvent paraître raisonnables sur un système ne le sont
pas sur tous Par exemple, les machines de l'ISTY ou de ens-info n'ont
généralement pas plus de 30 ou 40 utilisateurs simultanéments connectés,
mais on peut trouver des machines avec 500 (ou plus encore)
utilisateurs.

%%%%%%%%%%%%%%%%%%%%%%%%%%%%%%%%%%%%%%%%%%%%%%%%%%%%%%%%%%%%%%%%%%%%%%%%%%%%%%

\regle {Ce qui est vrai sur un système ne l'est pas forcément sur tous}

Par exemple, ce n'est pas parce que sur HP et Sun, les noms des
utilisateurs sont limités à 8 caractères, que c'est vrai sur tous les
autres systèmes.  Autre exemple :  ce n'est pas parce que tel fichier
est dans {\tt /etc} ({\tt /etc/utmp} par exemple) sur HP qu'il l'est sur
tous les Unix (sur FreeBSD, c'est le fichier {\tt /var/run/utmp}).

%%%%%%%%%%%%%%%%%%%%%%%%%%%%%%%%%%%%%%%%%%%%%%%%%%%%%%%%%%%%%%%%%%%%%%%%%%%%%%

\regle {Lisez les manuels}

Une règle importante pour rédiger des programmes portables est de bien lire
les manuels en ligne, surtout les petites lignes tout en bas sur HP...

%%%%%%%%%%%%%%%%%%%%%%%%%%%%%%%%%%%%%%%%%%%%%%%%%%%%%%%%%%%%%%%%%%%%%%%%%%%%%%

\regle {Discriminez les caractéristiques, par les architectures}

Réaliser des programmes portables impose parfois d'utiliser la
compilation conditionnelle (\verb|#ifdef|...).  Il faut éviter
de discriminer suivant les architectures (Sun ou HP par exemple), mais
il vaut mieux discriminer suivant des caractéristiques. Par exemple~:

\begin {itemize}
    \item définir le symbole {\tt UNAME} suivant que le champ de la
	structure {\tt utmp} contenant le nom d'utilisateur s'appelle
	{\tt ut\_user} ou {\tt ut\_name}

    \item définir le symbole {\tt HAS\_MACHIN} si la primitive {\tt
	machin()} est disponible

\end {itemize}

%%%%%%%%%%%%%%%%%%%%%%%%%%%%%%%%%%%%%%%%%%%%%%%%%%%%%%%%%%%%%%%%%%%%%%%%%%%%%%

\regle {Pas de cibles inutiles}

Si vous avez bien lu ce qui précède, il ne vous viendra pas à l'idée
d'avoir des cibles~:

\begin {quote}
{\tt monprog\_hp:} ... \\
\hspace* {10mm} actions pour compiler sur HP \\
{\tt monprog\_sun:} ... \\
\hspace* {10mm} actions pour compiler sur Sun
\end {quote}

... et c'est tant mieux !

%%%%%%%%%%%%%%%%%%%%%%%%%%%%%%%%%%%%%%%%%%%%%%%%%%%%%%%%%%%%%%%%%%%%%%%%%%%%%%

\regle {Ne prenez pas de décision en fonction de l'environnement de l'utilisateur}

N'utilisez pas les variables d'environnement du Shell, car elles sont
bien souvent personnelles.
Par exemple, beaucoup de projets, dans les années antérieures, ont utilisé
la variable
{\tt ARCH}~: cette variable était initialisée dans le {\tt .profile}
par défaut des étudiants. Le problème est que toute autre personne qui
n'avait pas ce fichier {\tt .profile}, en particulier le correcteur
du projet, était incapable de le compiler.

%%%%%%%%%%%%%%%%%%%%%%%%%%%%%%%%%%%%%%%%%%%%%%%%%%%%%%%%%%%%%%%%%%%%%%%%%%%%%%

\regle {Il n'y a qu'un seul fichier à modifier, c'est le Makefile}

Mettez tout ce qui est configurable (chemin d'accès à certains fichiers,
symboles du préprocesseur pour la portabilité, etc.) dans le fichier
{\tt Makefile}. Celui qui compile votre projet ne devrait jamais
avoir à modifier les sources de votre projet.

%%%%%%%%%%%%%%%%%%%%%%%%%%%%%%%%%%%%%%%%%%%%%%%%%%%%%%%%%%%%%%%%%%%%%%%%%%%%%%

\regle {Apprenez à distinguer ce qui est installé localement}

Un certain nombre de logiciels sont installés localement, à partir
de fichiers sources. Par exemple, les compilateurs de GNU, le système
graphique X-Window, etc. Apprenez à reconnaître ce qui est installé
localement de ce qui est disponible sur toutes les machines.
De plus, ce qui est installé localement ne l'est pas toujours au
même endroit sur toutes les machines, car c'est un choix laissé à
l'administrateur du système.

Par exemple, les fichiers d'inclusion (les \verb|.h|)  de X-Window
peuvent être dans tel ou tel répertoire. Vous devez pouvoir changer
cela très facilement (généralement, dans le {\tt Makefile}).

En particulier, ne mettez pas de nom absolu dans les {\tt \#include}.

%%%%%%%%%%%%%%%%%%%%%%%%%%%%%%%%%%%%%%%%%%%%%%%%%%%%%%%%%%%%%%%%%%%%%%%%%%%%%%

\regle {Mettez les cibles "all" et "clean" dans votre Makefile}

Les cibles traditionnelles {\tt all} et {\tt clean} sont devenues
une tradition bien utile.  Ne les oubliez pas.
