%
% Notice rapide d'utilisation des débogueurs
%
% Historique
%   1999/03/28 : pda : création
%

% \documentclass {article}
% 
%     \usepackage [9pt,a5,times] {my2e}
%     \usepackage {supertabular}
%     \setlength {\parskip} {3mm}
%     \setlength {\parindent} {0mm}
% 
% \begin {document}
% 
% \begin  {center}
%     \Large\bf Les débogeurs
% \end  {center}

Le débogueur est l'outil indispensable de mise au point
d'un programme. Cette courte notice a pour objet de présenter
les fonctions essentielles de deux débogueurs, à savoir
{\tt xdb} sur HP et {\tt gdb} (le débogueur de GNU, qui
fonctionne aussi sur HP). On se référera aux documentations
pour plus de détails.

\section {Introduction}

Un débogueur permet~:

\begin {itemize}
    \item de contrôler l'exécution d'un programme~:
	exécution en mode <<~pas à pas~>>, pose de
	<<~points d'arrêt~>>, affichage de
	variables, etc.
    \item d'analyser les raisons de la terminaison
	brutale d'un programme, grâce au fichier
	{\tt core} généré~: affichage de l'instruction
	qui a causé la terminaison, des variables
	au moment où le programme s'est arrêté, etc.
\end {itemize}

Pour pouvoir utiliser un débogueur, il faut procéder
à la compilation et à l'édition de liens du programme
avec l'option {\tt -g}. Cette option ajoute
au fichier exécutable des informations permettant au
débogueur de retrouver les adresses des instructions
et des variables.

Les débogueurs sous Unix fonctionnent de manière similaire.
Pour contrôler l'exécution d'un programme, le processus
exécutant le débogueur génère un processus fils pour le
programme à déboguer, puis en contrôle\footnote {Par le
biais de la primitive système {\tt ptrace}.} son
exécution.


\section {xdb}

{\renewcommand {\arraystretch} {1.05}
    % \tablehead {\hline \bf Commande & \bf Signification \\ \hline}
    % \tabletail {\hline}
%\begin {supertabular} {|l|p{90mm}|}
\begin {tabular} {|l|p{90mm}|}
    \hline \bf Commande & \bf Signification \\ \hline
p exp		& affiche l'expression (voir plus bas) \\
p exp$\backslash$format		& affiche l'expression suivant le format \\
p exp?format	& affiche l'adresse de l'expression suivant le format \\
    \hline
v fonction	& affiche la fonction source \\
v ligne		& affiche les lignes autour de la ligne spécifiée \\
v fichier	& affiche les premières lignes du fichier source \\
v fichier:ligne	& affiche las lignes autour de la ligne spécifiée \\
v		& affiche les lignes suivantes \\
+ [lignes]	& déplace la ligne courante \\
- [lignes]	& déplace la ligne courante \\
/ [chaîne]	& cherche la chaîne dans le fichier \\
? [chaîne]	& cherche la chaîne dans le fichier (en remontant) \\
    \hline
t		& affiche l'empilement des fonctions dans la pile \\
T		& idem t, avec affichage des variables locales \\
up [nb]		& remonte de nb niveaux dans la pile \\
down [nb]	& descend de nb niveaux dans la pile \\
V [profondeur]	& affiche le source correspondant à la fonction à la profondeur spécifiée dans la pile \\
l [fonction]	& affiche les paramètres et variables locales de la fonction \\
    \hline
r [args]	& lance le programme avec les arguments spécifiés \\
R		& lance un nouveau programme sans arguments \\
k		& termine le programme courant \\
    \hline
c		& continue l'exécution jusqu'au prochain point d'arrêt \\
c fonction	& continue l'exécution jusqu'à la fonction \\
c ligne		& continue l'exécution jusqu'à la ligne spécifiée \\
c fichier:ligne	& continue l'exécution jusqu'à la ligne spécifiée \\
s [nb]		& exécute nb instructions \\
S [nb]		& exécute nb instructions, en comptant les appels de de fonctions comme de simples instructions \\
    \hline
lb		& liste tous les points d'arrêts \\
b		& pose un point d'arrêt sur la ligne courante \\
b fonction	& pose un point d'arrêt sur la première instruction de la fonction \\
b ligne		& pose un point d'arrêt sur la ligne \\
b fichier:ligne	& pose un point d'arrêt sur la ligne \\
db [numéro]	& supprime le point d'arrêt spécifié \\
db *		& supprime tous les points d'arrêts \\
bp		& pose un point d'arrêt sur tous les débuts de fonctions \\
bpx		& pose un point d'arrêt sur toutes les sorties de fonctions \\
bpt		& trace toutes les entrées et sorties de fonctions \\
Dp		& supprime tous les points d'arrêts de fonctions \\
Dpx		& supprime tous les points d'arrêts de sortie de fonctions \\
Dpt		& arrête la trace des fonctions \\
    \hline
w [size]	& change la taille de la fenêtre source \\
D "répertoire"	& ajoute un répertoire pour la localisation des fichiers source \\
\hline
\end {tabular}
%\end {supertabular}
}

Pour afficher des variables, il faut utiliser la commande <<~p~>>.
L'expression peut être une expression en C quelconque.
Par exemple~: \verb|p *(x->fg->d.tab [i])+3|. Le format permet
de préciser le type de donnée à afficher. Quelques formats parmi
les plus courants sont cités ci-dessous~:

\begin {tabular} {|c|l|} \hline
    \bf Format & \bf Signification \\ \hline
    n & format normal \\
    d & décimal \\
    u & décimal non signé \\
    x & hexadécimal \\
    z & binaire \\
    c & caractère \\
    s & chaîne de caractères \\
    t & affiche le type de l'expression (et non la valeur) \\
    \hline
\end {tabular}

Pour modifier une variable, il suffit d'utiliser une expression
d'affectation. Par exemple~: \verb|p y=5|.



\section {gdb}


{\renewcommand {\arraystretch} {1.1}
    \tablehead {\hline \bf Commande & \bf Signification \\ \hline}
    \tabletail {\hline}
\begin {supertabular} {|l|p{80mm}|}
print/p exp		& affiche l'expression (voir plus bas) \\
print/p /format exp	& affiche l'expression suivant le format \\
x adresse		& affiche la mémoire à l'adresse spécifiée \\
    \hline
list/l fonction	& affiche la fonction source \\
list/l ligne		& affiche les lignes autour de la ligne spécifiée \\
list/l fichier	& affiche les premières lignes du fichier source \\
list/l fichier:ligne	& affiche las lignes autour de la ligne spécifiée \\
list/l		& affiche les lignes suivantes \\
search [chaîne]	& cherche la chaîne dans le fichier \\
reverse-search [chaîne]	& cherche la chaîne dans le fichier (en remontant) \\
    \hline
backtrace/bt	& affiche l'empilement des fonctions dans la pile \\
up [nb]		& remonte de nb niveaux dans la pile \\
down [nb]	& descend de nb niveaux dans la pile \\
frame/f [profondeur]	& sélectionne la fonction à la profondeur spécifiée dans la pile \\
info locals	& affiche les variables locales de la fonction \\
    \hline
run [args]	& lance le programme avec les arguments spécifiés \\
kill		& termine le programme courant \\
    \hline
continue/c		& continue l'exécution jusqu'au prochain point d'arrêt \\
continue/c fonction	& continue l'exécution jusqu'à la fonction \\
continue/c ligne		& continue l'exécution jusqu'à la ligne spécifiée \\
continue/c fichier:ligne	& continue l'exécution jusqu'à la ligne spécifiée \\
step/s [nb]		& exécute nb instructions \\
next/n [nb]		& exécute nb instructions, en comptant les appels de de fonctions comme de simples instructions \\
finish		& continue l'exécution jusqu'à la fin de la fonction courante \\
    \hline
info break	& liste tous les points d'arrêts \\
break/b		& pose un point d'arrêt sur la ligne courante \\
break/b fonction	& pose un point d'arrêt sur la première instruction de la fonction \\
break/b ligne		& pose un point d'arrêt sur la ligne \\
break/b fichier:ligne	& pose un point d'arrêt sur la ligne \\
rbreak regexp	& pose un point d'arrêt sur toutes les fonctions correspondant à l'expression régulière \\
clear		& supprime le point d'arrêt sur la ligne courante \\
clear fonction	& supprime le point d'arrêt sur la première instruction de la fonction \\
clear ligne		& supprime le point d'arrêt sur la première instruction suivant la ligne \\
clear fichier:ligne	& supprime le point d'arrêt sur la première instruction suivant la ligne \\
delete/d 	& supprime tous les points d'arrêts \\
    \hline
dir "répertoire"	& ajoute un répertoire pour la localisation des fichiers source \\
\end {supertabular}
}

Pour afficher des variables, il faut utiliser la commande <<~p~>>.
L'expression peut être une expression en C quelconque.
Par exemple~: \verb|p *(x->fg->d.tab [i])+3|. Le format permet
de préciser le type de donnée à afficher. Quelques formats parmi
les plus courants sont cités ci-dessous~:

\begin {tabular} {|c|l|} \hline
    \bf Format & \bf Signification \\ \hline
    d & décimal \\
    u & décimal non signé \\
    x & hexadécimal \\
    t & binaire \\
    c & caractère \\
    s & chaîne de caractères \\
    \hline
\end {tabular}

Pour modifier une variable, il suffit d'utiliser une expression
d'affectation. Par exemple~: \verb|p y=5|.




% \end {document}
