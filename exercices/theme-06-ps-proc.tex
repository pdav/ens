%%%%%%%%%%%%%%%%%%%%%%%%%%%%%%%%%%%%%%%%%%%%%%%%%%%%%%%%%%%%%%%%%%%%%%%%%%%%%%%
% PRIMITIVES DE GESTION DE PROCESSUS
%
% Historique
%   1993/01/31 : pda : création
%   1996/01/06 : pda : séparation en deux TD
%   1996/09/10 : pda : réunion en un seul thème
%%%%%%%%%%%%%%%%%%%%%%%%%%%%%%%%%%%%%%%%%%%%%%%%%%%%%%%%%%%%%%%%%%%%%%%%%%%%%%%

\td {Système~: Gestion des processus}

% \but
% 
% L'objet de ce TD est l'étude des primitives système de
% gestion des processus.


\question

Écrivez un programme générant un processus fils avec la primitive
système {\tt fork}.

\begin {itemize}
    \item le processus fils doit afficher son numéro ({\em pid}) ainsi
	que le numéro du père à l'aide des primitives système {\tt
	getpid} et {\tt getppid}, puis sort (primitive {\tt exit}) avec
	un code de retour égal au dernier chiffre du {\em pid}.

    \item le processus père, quant à lui, affiche le {\em pid} du fils,
	puis attend sa terminaison (primitive {\tt wait}) et affiche son
	code de retour.

\end {itemize}


\question

Écrivez un programme qui lance {\em n} processus fils dans une
première étape puis, dans une deuxième étape, attend leur terminaison
à l'aide de la primitive {\tt wait}.
\`A chaque fois qu'un processus se termine,
le père affiche son numéro ({\em pid}) et son code de retour.


\question

Écrivez un programme ayant la syntaxe suivante~:

\vspace* {-3mm}
\begin {quote}
{\tt matproc $n$ $m$}
\end {quote}

L'action de ce programme doit être de générer $n$ processus, chacun
d'entre-eux devant générer $n$ processus à son tour, et ainsi de suite
jusqu'à $m$ niveaux.

Combien de processus sont générés au total ?


\question

Écrivez un programme qui~:

\begin {enumerate}
    \item lance la commande {\tt ls} sur un répertoire passé en
        paramètre,
    \item redirige la sortie standard de {\tt ls} sur {\tt /dev/null},
    \item affiche le temps pris par la commande {\tt ls} (primitive
	système {\tt times}).
\end {enumerate}


\question

Écrivez un programme qui~:

\begin {enumerate}
    \item attende une ligne (une commande) sur l'entrée standard (vous
	pouvez utiliser la fonction de librairie {\tt fgets})~;

    \item la décompose en mots séparés par des espaces~;

    \item recherche le premier mot dans le {\tt PATH}~;

    \item exécute cette commande par le biais de la primitive système
	{\tt execv}~;
    
    \item revienne au point 1.

\end {enumerate}

Félicitations, vous avez écrit un {\em shell}~!


\question

Reprenez le programme de l'exercice précédent et ajoutez un traitement
spécial dans le cas où la commande est terminée par le mot \verb:&:.


