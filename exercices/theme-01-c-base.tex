%%%%%%%%%%%%%%%%%%%%%%%%%%%%%%%%%%%%%%%%%%%%%%%%%%%%%%%%%%%%%%%%%%%%%%%%%%%%%%%
% LANGAGE C : ELEMENTS DE BASE
%
% Historique
%   1993/01/24 : pda : création
%   1995/10/04 : pda : adaptation pour 95/96
%   1997/08/26 : pda : ajout de l'exercice lire_16
%   1997/08/26 : pda : ajout de l'exercice ecrire_2
%%%%%%%%%%%%%%%%%%%%%%%%%%%%%%%%%%%%%%%%%%%%%%%%%%%%%%%%%%%%%%%%%%%%%%%%%%%%%%%

\td {Langage~C~: Éléments de base}

% \but
% 
% Les exercices de ce thème ont pour but l'assimilation des éléments de
% base du langage~C~:  déclarations des variables, structures de contrôle,
% structures, chaînes de caractères, et utilisation d'un jeu minimum de
% fonctions d'entrées/sorties.


\question
%%%% Dependances: for, printf

Écrivez un programme qui affiche à l'écran les valeurs de $2^i$ ($0
\leq i < 10$) en utilisant la fonction {\tt printf}.


\question
%%%% Dependances: getchar, while, EOF, constantes char

Écrivez un programme qui lit des caractères (avec la fonction {\tt
getchar}) sur l'entrée standard et compte le nombre d'occurrences de
chaque lettre (on ne distinguera pas les minuscules des majuscules).

\question
%%%% Dependances: getchar, while, EOF

Écrivez un programme qui lit des phrases (c'est-à-dire des suites de
caractères quelconques) sur l'entrée standard et compte le nombre de
mots (c'est-à-dire les suites de caractères composées exclusivement de
lettres minuscules ou majuscules) qui s'y trouvent.


\question
%%%% Dependances: getchar, while, EOF, \n

Modifiez le programme de l'exercice précédent pour afficher seulement
les mots (et pas les chiffres, les signes de ponctuation, etc.), un par
ligne.


\question
%%%% Dependances: getchar, putchar, tableaux, (chaines), modulo

Avec les fonctions {\tt getchar} et {\tt putchar}, écrivez un programme
qui affiche les 10 dernières lignes lues sur l'entrée standard.


\question
%%%% Dependances: chaînes, fgets, puts, printf, strncpy, strlen

Écrivez un programme qui lit une chaîne de caractères sur l'entrée
standard (avec {\tt fgets}), la recopie (avec {\tt strncpy}) dans une
autre chaîne, puis affiche la nouvelle chaîne (avec {\tt puts}) et
sa longueur (avec {\tt strlen} et {\tt printf}).


\question
%%%% Dependances: fgets, puts, chaînes
    \label {strbrk}

Écrivez un programme qui lit deux chaînes, puis cherche si la deuxième
chaîne fait partie de la première, et affiche un message en conséquence.
Vous n'utiliserez pas de fonction de librairie autre que {\tt fgets} et
{\tt puts}.


\question
%%%% Dependances: chaînes, structures, printf

On désire avoir un programme qui lit sur l'entrée standard un nom de mois
(entre \verb:"janvier":  et \verb:"decembre":), et qui affiche son
numéro (entre 1 et 12) ainsi que le nombre de jours dans ce
mois\footnote {On ne tient pas compte du cas du mois de février lors
des années bissextiles.}. Si le mois n'est pas valide,
il doit afficher un message d'erreur.

Pour cela, on propose de placer les mois valides ainsi que le nombre de
jours correspondant, dans une structure~:

\begin {quote}
\small
\begin {verbatim}
struct mois
{
    char nom [9 + 1] ;     /* nom du mois en clair */
    int  jours ;           /* nombre de jours dans le mois */
} ;
\end{verbatim}
\end {quote}

Déclarez un tableau constant (initialisé) de 12 structures de cette
forme pour mémoriser les 12 mois possibles et écrivez le programme.


\question
%%%% Dependances: chaînes, switch
    \label {backslash}

Les chaînes de caractères et les constantes de type caractère en
langage~C peuvent contenir les éléments suivants~:

\begin {itemize}
    \item caractères ``normaux'', c'est à dire tous les caractères de
	code compris entre 32 et 126~;
    \item caractères de contrôle ``classiques''
	(\verb:\n:, \verb:\r:, \verb:\t:, et \verb:\b:)~;
    \item caractères non représentables tels quels
	(\verb:\\:, \verb:\': et \verb:\":)~;
    \item autres caractères, dont la valeur numérique est comprise entre
	0 et 31 ou supérieure à 127. Dans ce cas, la représentation est
	un antislash (\verb:\:) suivi du code octal du caractère.
\end {itemize}

Écrivez un programme qui lit une chaîne de caractères contenant
éventuellement des caractères spéciaux, puis place dans une deuxième
chaîne la représentation~C (c'est à dire avec le caractère \verb:\t:
remplacé par les deux caractères \verb:\: et \verb:t: par exemple).


\question
%%%% Dependances: chaînes, switch, base, ASCII

Écrivez le programme inverse de l'exercice~\ref {backslash}, pour
transformer une chaîne de caractère contenant éventuellement des
caractères spéciaux en représentation~C, en chaîne de caractères
contenant les caractères traduits (c'est à dire avec les deux caractères
\verb:\:  et \verb:t:  remplacés par le caractère \verb:\t:  par
exemple).


\question
%%%% Dependances: cpp

Le programmeur qui a écrit la définition ci-dessous aura certainement
quelques problèmes lorsqu'il l'utilisera.
Expliquez ces problèmes et
aidez ce pauvre programmeur à améliorer sa définition.

\begin {quote}
\begin {verbatim}
#define carre(x) x*x
\end{verbatim}
\end {quote}


\question
%%%% Dependances: cpp, fonctions, passage par valeur

Expliquez les différences entre votre version corrigée de l'exercice
précédent et la définition ci-dessous~:

\begin {quote}
\begin {verbatim}
int carre (int x) { return x * x ; }
\end{verbatim}
\end {quote}



\question
%%%% Dependances: passage par valeur, operateur --

Pourquoi la fonction ci-après peut donner des résultats faux~?

\begin {quote}
\small
\begin {verbatim}
int factorielle (int n)
{
    int f ;
    if (n <= 1) f = 1 ; else f = n * factorielle (--n) ;
    return f ;
}
\end{verbatim}
\end {quote}


\question
%%%% Dependances: passage par valeur, ASCII, base, opérateurs binaires
    \label {lire16}

Écrivez la fonction :
\begin {quote}
    \verb:unsigned int lire_16 (void):
\end {quote}
qui lit avec
{\tt fgets} une chaîne contenant des chiffres hexadécimaux, et renvoie la
valeur convertie.
Vous n'utiliserez pas les opérateurs \verb:*:, \verb:/:  et \verb:%:.


\question
%%%% Dependances: passage par valeur, ASCII, base, opérateurs binaires

Écrivez la fonction :
\begin {quote}
    \verb:void ecrire_2 (unsigned int):
\end {quote}
qui prend un
nombre en paramètre, et en affiche la représentation binaire sur la
sortie standard.
Comme pour l'exercice~\ref {lire16}, vous n'utiliserez pas les opérateurs
\verb:*:, \verb:/:  et \verb:%:.


