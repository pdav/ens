%%%%%%%%%%%%%%%%%%%%%%%%%%%%%%%%%%%%%%%%%%%%%%%%%%%%%%%%%%%%%%%%%%%%%%%%%%%%%%%
% SIGNAUX V7, SIGNAUX POSIX
%
% Historique
%   1993/01/31 : pda : création
%   1995/05/02 : pda : ajout des signaux POSIX
%   1996/02/28 : pda : séparation des tubes
%   1996/09/10 : pda : séparation de l'environnement
%   1996/09/10 : pda : réunion des signaux POSIX
%   2007/04/19 : pda : séparation exercices 1 et 2
%   2007/04/19 : pda : rédaction exercice envoyer/reception
%   2013/09/23 : vincent : ajout exercice 5 (imbrication de signaux)
%   2016/03/29 : pda : ajout exercices 1 et 2 (SIGINT x 5, SIGQUIT)
%%%%%%%%%%%%%%%%%%%%%%%%%%%%%%%%%%%%%%%%%%%%%%%%%%%%%%%%%%%%%%%%%%%%%%%%%%%%%%%

\td {Système~: Signaux}

% \but
% 
% L'objet de ce TD est l'étude des signaux V7 et POSIX.


\question

À l'aide des primitives V7, écrivez un programme qui attend 5 fois
le signal \texttt {SIGINT}. À chaque réception du signal, votre
programme doit afficher la valeur du compteur, et au bout de 5 fois,
il doit s'arrêter. L'incrémentation du compteur ne doit pas être
faite dans la fonction \texttt {main}.


\question

Activez la génération des fichiers «~\texttt {core}~»\footnote
{Avec \texttt {bash}, utilisez la commande «~\texttt {ulimit -c
unlimited}~».}.  En utilisant le programme de l'exercice
précédent, utilisez le signal \texttt {SIGQUIT} (obtenu avec \texttt
{\^{}\textbackslash} dans un shell) pour interrompre le processus et
générer un fichier «~\texttt {core}~».

Utilisez \texttt {gdb} pour afficher l'endroit où le programme s'est
arrêté (commandes \texttt {where}, \texttt {up}/\texttt {down} et
\texttt {list}) et la valeur du compteur (command \texttt {print}).


\question

À l'aide des primitives V7, écrivez un programme qui attend l'arrivée
d'un signal (n'importe lequel), affiche sa signification (par
exemple~:  {\tt illegal instruction} pour {\tt SIGILL}) puis se
termine.


\question

Modifiez le programme de l'exercice précédent pour traiter plusieurs
signaux consécutifs~: votre programme ne doit pas se terminer après
l'arrivée d'un signal.


\question

Écrivez un programme composé de deux processus.  Le processus père
génère un processus fils qui doit exécuter une fonction {\tt traite}
toutes les secondes, qui ne fait qu'afficher un message.  Au bout d'une
minute, le processus père affiche un message et prévient le fils qu'il
doit s'arrêter en lui envoyant le signal \texttt {SIGUSR1}. Lorsque
le processus fils reçoit l'ordre du père, il affiche un message et
s'arrête effectivement, provoquant alors la terminaison du père. Vous
utiliserez les primitives V7, avec le signal \texttt {SIGALRM} pour tout
ce qui est temporisation.


\question

Écrivez un programme composé d'une boucle sans fin qui
incrémente un compteur. Lorsque l'utilisateur appuie sur la touche
d'interruption (signal {\tt SIGINT}), le programme sauve la date en
clair et la valeur courante du compteur dans un fichier,
à la suite de ce qui s'y trouve déjà.
Lorsque le signal {\tt SIGTERM} est reçu, le programme écrit
le mot {\tt fin} dans le fichier, puis se termine. Vous utiliserez les
primitives POSIX.


\question

Écrivez un programme composé d'une boucle sans fin qui affiche le
message « traitement normal » à chaque seconde. Lorsqu'il reçoit un
signal {\tt SIGINT}, il exécute une fonction qui affiche le message
« traitement du signal de niveau 1 », puis attend une seconde, cinq
fois de suite. Lorsqu'il reçoit une nouvelle fois le signal \texttt
{SIGINT}, il augmente la valeur du niveau affiché, puis revient au
niveau précédent au bout de ses cinq affichages. Vous pouvez afficher
des tabulations correspondant au niveau de traitement courant pour plus
de lisibilité. Vous utiliserez les primitives POSIX.


\question

On désire simuler un mécanisme matériel comparable à un coupleur série
à l'aide des signaux POSIX. Un 0 (zéro) est matérialisé par le signal
{\tt SIGUSR1}, un 1 (un) est matérialisé par le signal {\tt SIGUSR2}.
Un octet est transmis par une succession de 8 bits (0 ou 1).
À chaque fois qu'il reçoit un bit, le récepteur doit envoyer en
retour un acquittement (signal {\tt SIGUSR1}) pour prévenir l'émetteur
qu'il peut passer au bit suivant.

Écrivez les fonctions suivantes~:

\begin {enumerate}
    \item \verb:void envoyer (pid_t recepteur, int octet):

	Cette fonction envoie les 8 bits constituant un octet au
	processus désigné.

    \item \verb:void preparer_reception (pid_t emetteur):

	Cette fonction prépare le récepteur à recevoir un octet.

    \item \verb:void int recevoir (void):

	Cette fonction attend que suffisamment de bits soient reçus
	pour constituer un octet,  et renvoie alors la valeur reçue.

\end {enumerate}


\question

En utilisant un protocole similaire à celui de l'exercice précédent,
on désire simuler le fonctionnement d'un tube, c'est à dire réaliser
les fonctions suivantes~:

\begin {enumerate}
    \item \verb:void preparer_tube (void):

	Cette fonction initialise les structures de données associées
	au tube.

    \item \verb:void processus_tube (pid_t autre, int sens):

	Connaissant le \textit {pid} de l'autre processus, cette
	fonction initialise l'accès au tube pour une lecture (sens
	= 1) ou pour une écriture (sens = 2).

    \item \verb:void fermer_tube (void):

	Cette fonction ferme le tube.

    \item \verb:void ecrire_tube (void *buffer, int longueur):

	Cette fonction écrit dans le tube les données spécifiées.

    \item \verb:int lire_tube (void *buffer, int longueur):

	Cette fonction extrait du tube les données et les renvoie
	dans le buffer. Le nombre d'octets lus est renvoyé (0 en
	fin de tube).

\end {enumerate}


