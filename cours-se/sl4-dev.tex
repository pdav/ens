\def\inc{inc4-dev}

\titreA {Pilotes de périphériques}

%%%%%%%%%%%%%%%%%%%%%%%%%%%%%%%%%%%%%%%%%%%%%%%%%%%%%%%%%%%%%%%%%%%%%%%%%%%%%%
% Mécanismes matériels
%%%%%%%%%%%%%%%%%%%%%%%%%%%%%%%%%%%%%%%%%%%%%%%%%%%%%%%%%%%%%%%%%%%%%%%%%%%%%%

\titreB {Mécanismes matériels}

\begin {frame} {Mécanismes matériels}
    Rappel : architectures matérielles

    \begin {center}
	\includegraphics [width=.7\linewidth] {\inc/arch-now}
    \end {center}

    Comment le processeur interagit avec les périphériques ?
\end {frame}

\begin {frame} {Mécanismes matériels}
    Vision simplifiée : les périphériques sont sur le bus système
    \begin {center}
	\includegraphics [width=.4\linewidth] {\inc/bus}
    \end {center}
    \begin {itemize}
	\item contrôleur de périphérique connecté au bus
	    \begin {itemize}
		\item il reconnaît une plage d'adresses spécifique
		\item il surveille le bus de contrôle
	    \end {itemize}
	\item le processus lit ou écrit a des adresses spécifiques
	    pour interroger le contrôleur ou lui transmettre des requêtes
	\item de plus, le contrôleur peut émettre une interruption
	    sur le bus de contrôle
    \end {itemize}
\end {frame}

\begin {frame} {Mécanismes matériels}
    Deux mécanismes fondamentaux pour lire ou écrire à des adresses
    spécifiques :

    \begin {itemize}
	\item instructions spéciales d'entrées/sorties
	    \begin {itemize}
		\item exemple sur i386 : instructions IN et OUT
		\item instructions privilégiées
		\item tous les processeurs n'ont pas ce type d'instructions
		\item mécanisme obsolète, mais toujours utilisé (ah le PC...)
	    \end {itemize}
	\item entrées/sorties dans l'adressage « normal »
	    \begin {itemize}
		\item utilisation de l'instruction MOVE classique
		\item comme pour la mémoire
		\item mais pour des adresses spécifiques
		\item adresses interdites aux processus (cf chapitre suivant)
		\item utilisable sur tous les processeurs actuels
		\item avantage : utilisable en C (pas d'assembleur...)
	    \end {itemize}
    \end {itemize}

    L'utilisation de IN et OUT plutôt que l'adressage « normal »
    se traduit par des signaux distincts sur le bus de contrôle.
\end {frame}

\begin {frame} {Exemple -- Clavier PC-AT}
    \begin {itemize}
	\item deux adresses reconnues par le contrôleur : 0x60 et 0x64
	\item significations différentes entre lecture et écriture
	    \begin {itemize}
		\item \textit {input buffer} : en écriture, permet
		    au processeur d'envoyer une information au contrôleur
		\item \textit {output buffer} : en lecture, transfert d'un
		    octet du contrôleur vers le processeur
		\item \textit {control register} : en écriture, permet
		    au processeur d'envoyer une commande au contrôleur
		\item \textit {status register} : en lecture, indique
		    via 8 bits l'état du contrôleur
	    \end {itemize}
    \end {itemize}
    \begin {center}
	\includegraphics [width=.7\linewidth] {\inc/kbd-ctrl}
    \end {center}
\end {frame}

\begin {frame} {Exemple -- Clavier PC-AT}
    Exemple 1 : lire une touche sur le clavier

    \lstinputlisting [basicstyle=\fD\lstmonstyle] {\inc/kbd-read.s}
\end {frame}

\begin {frame} {Exemple -- Clavier PC-AT}
    Exemple 2 : allumer les LED CapsLock et NumLock du clavier

    \lstinputlisting [basicstyle=\fD\lstmonstyle] {\inc/kbd-led.s}
\end {frame}

\begin {frame} {Mécanismes matériels -- Interruptions}
    Certains périphériques permettent d'interrompre le processeur

    \begin {itemize}
	\item capacité contrôlée par le \textit {control register}
	\item la lecture du \textit {status register} par le processeur
	    suffit pour stopper l'état « interruptif »
    \end {itemize}

    \vspace* {3mm}

    Note : sur l'architecture PC, un composant spécifique gère les
    interruptions

    \begin {itemize}
	\item PIC = \textit {Programmable Interrupt Controller}
    \end {itemize}
\end {frame}

\begin {frame} {Mécanismes matériels -- DMA}
    Le contrôleur de clavier ne transfère qu'un code de touche à
    la fois \implique périphérique bas débit

    \vspace* {3mm}

    Certains périphériques fonctionnent à plus haut débit :

    \begin {itemize}
	\item contrôleur disque
	\item carte réseau
    \end {itemize}

    \implique DMA (\textit {Direct Memory Access})

    \begin {enumerate}
	\item le processeur envoie une commande au périphérique en
	    indiquant une adresse de transfert (lecture ou écriture)
	    des données

	\item le contrôleur de périphérique traite la requête en
	    accédant directement à la mémoire
	    \begin {itemize}
		\item processeur complètement déchargé
	    \end {itemize}

	\item lorsque le contrôleur a terminé, il génère une interruption

    \end {enumerate}
\end {frame}

%%%%%%%%%%%%%%%%%%%%%%%%%%%%%%%%%%%%%%%%%%%%%%%%%%%%%%%%%%%%%%%%%%%%%%%%%%%%%%
% Pilotes de périphériques
%%%%%%%%%%%%%%%%%%%%%%%%%%%%%%%%%%%%%%%%%%%%%%%%%%%%%%%%%%%%%%%%%%%%%%%%%%%%%%

\titreB {Pilotes de périphériques}

\begin {frame} {Pilotes de périphériques}
    Rappels sur noyau Unix originel :

    \begin {itemize}
	\item Périphériques identifiés par un numéro composé de :
	    \begin {itemize}
		\item majeur : identifie le pilote
		\item mineur : identifie le périphérique géré par le pilote
	    \end {itemize}
	\item Inode des fichiers dans \code {/dev}
	    \begin {itemize}
		\item type : périphérique en mode bloc, ou en mode caractère
		\item rdev : numéro du périphérique
	    \end {itemize}
	\item Pilote : ensemble de fonctions intégrées au noyau
	    \begin {itemize}
		\item mode bloc : \code {open}, \code {close}, \code
		    {strategy}, traitement d'interruption

		\item mode caractère : \code {open}, \code {close}, \code
		    {read}, \code {write}, \code {ioctl}, traitement
		    d'interruption
	    \end {itemize}
	\item Tableaux de pointeurs sur des fonctions, indexés par le majeur
	    \begin {itemize}
		\item mode bloc : \code {bdevsw [maj].open}
		\item mode caractère : \code {cdevsw [maj].read}
	    \end {itemize}
    \end {itemize}
\end {frame}

\begin {frame} {Pilotes de périphériques}
    Cheminement d'une requête :

    \begin {center}
	\includegraphics [width=1\linewidth] {\inc/cdevsw}
    \end {center}
\end {frame}

\begin {frame} {Pilotes de périphériques -- Carte réseau}
    Autre type de pilote : les pilotes de \textbf {cartes réseau}

    \begin {itemize}
	\item pilotes spécialisés pour les protocoles réseaux
	\item pas de fichier spécial périphérique dans \code {/dev}
	    \begin {itemize}
		\item pas d'« ouverture » de carte réseau...
		\item ... mais une connexion à une adresse IP (p. ex.)
		\item \implique primitives spécialisées (cf semestre 5)
	    \end {itemize}
	\item pas les mêmes fonctions que les pilotes « classiques »
	    \begin {itemize}
		\item accès à la carte géré par les protocoles réseaux
		\item pas d'interaction « directe » avec les processus
	    \end {itemize}
	\item type de pilote non traité ici
    \end {itemize}
\end {frame}

\begin {frame} {Évolutions}
    Pilote : ensemble de fonctions intégrées au noyau

    \begin {itemize}
	\item Évolution vers un monde plus dynamique
	    \begin {itemize}
		\item brancher et débrancher « à chaud » les périphériques
		\item reconnaître les périphériques
		    \begin {itemize}
			\item Exemple : clavier USB français
			\item Exemple : moniteur avec une résolution
			    de ...
		    \end {itemize}
	    \end {itemize}
	    \implique évolution des fonctions requises pour un pilote

	\item Exemple : fonctions supplémentaires sur FreeBSD
	    \ctableau {\fD} {|l|l|} {
		\code {identify} & reconnaître le périphérique \\
		\code {probe} & vérifier si le périphérique est connecté \\
		\code {attach} & insérer le périphérique dans
		    les structures de données du noyau \\
		\code {detach} & retirer le périphérique \\
		\code {shutdown} & prépare le retrait \\
		\code {suspend} & appelée lors de la mise en veille \\
		\code {resume} & appelée lors de la reprise \\
	    }
    \end {itemize}
\end {frame}

\begin {frame} {Évolutions}
    Monde plus complexe :
    \begin {itemize}
	\item un même type de périphérique peut être sur des bus différents
	    selon son interface
	    \begin {itemize}
		\item Exemples :
		    \begin {itemize}
			\item disque : sur un bus IDE,
			    SAS, SATA, PCIe, USB, etc.
			\item audio : sur le bus système, PCI, etc.
		    \end {itemize}
	    \end {itemize}
	\item \implique Évolution vers des pilotes plus spécialisés
    \end {itemize}
\end {frame}

\begin {frame} {Évolutions}
    Exemple (imaginaire) :
    \begin {center}
	\includegraphics [width=.7\linewidth] {\inc/spec}
    \end {center}

    \begin {itemize}
	\item Empilement de pilotes :
	    \begin {itemize}
		\item interfaces spécifiques entre pilotes
		\item un des pilotes doit présenter une interface « classique »
	    \end {itemize}
	\item Pilotes de cartes réseau :
	    \begin {itemize}
		\item exemples de pilotes spécifiques sans interface
		    directe avec les primitives systèmes
	    \end {itemize}
    \end {itemize}
\end {frame}

%%%%%%%%%%%%%%%%%%%%%%%%%%%%%%%%%%%%%%%%%%%%%%%%%%%%%%%%%%%%%%%%%%%%%%%%%%%%%%
% Répartition noyau/processus
%%%%%%%%%%%%%%%%%%%%%%%%%%%%%%%%%%%%%%%%%%%%%%%%%%%%%%%%%%%%%%%%%%%%%%%%%%%%%%

\titreB {Répartition noyau/processus}

\begin {frame} {Répartition noyau/processus}
    Rôle du pilote : communiquer avec un périphérique...
    \\
    ... mais échanger des octets $\neq$ contrôler le périphérique

    \vspace* {3mm}

    Exemples :

    \begin {itemize}
	\item le pilote de terminal permet d'envoyer des octets

	    \begin {itemize}
		\item effacer l'écran, contrôler le curseur, etc.
		    \implique envoi de séquences d'échappement
		\item séquences différentes suivant les terminaux
		    \implique base \code {termcap} ou \code
		    {terminfo}
		\item applications \code {vi}, \code {more}, etc :
		    exploitent cette base

	    \end {itemize}

	\item le pilote de GPS permet d'échanger des octets

	    \begin {itemize}
		\item composant GPS : envoi périodique de
		    «~trames~» NMEA\footnote {\fE \textit
		    {National Marine Electronics Association}}

		\item {\fE \code {\$GPGGA,161229.487,3723.2475,N,12158.3416,W,1,07,1.0,9.0,M,-34.2,M,,0000*18}}

		\item l'application doit s'adapter à chaque composant
		    GPS et «~comprendre~» son protocole

	    \end {itemize}
    \end {itemize}
\end {frame}

\begin {frame} {Répartition noyau/processus}

    Gestion des imprimantes : plusieurs acteurs

    \begin {center}
	\includegraphics [width=1\linewidth] {\inc/impr}
    \end {center}
\end {frame}

\begin {frame} {Répartition noyau/processus}
    Gestion des imprimantes : plusieurs acteurs

    \begin {itemize}
	\item application : adapte les données pour l'imprimante

	    \begin {itemize}
		\item nécessite un paramétrage de l'application
		    et des programmes d'adaptation
		    \begin {itemize}
			\item exemple : conversion en
			    PostScript\texttrademark{}
			    ou en PCL\texttrademark
			\item appelé parfois à tort «~pilote d'imprimante~»
		    \end {itemize}
	    \end {itemize}

	\item spooler : reçoit les travaux pour l'imprimante, les met
	    en file d'attente et les envoie ensuite sur le périphérique

	    \begin {itemize}
		\item le spooler ne gère pas le contenu des fichiers
		\item pas d'adaptation pour l'imprimante
		\item généralement une file d'attente par imprimante
	    \end {itemize}

	\item noyau : traite les appels à \code {write} et les
	    redirige vers la couche réseau ou l'imprimante USB

    \end {itemize}
\end {frame}
