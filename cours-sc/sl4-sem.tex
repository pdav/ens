%
% Historique
%   2014/08/30 : pda : conception
%

\def\inc{inc4-sem}

\titreA {Sémaphores}

%%%%%%%%%%%%%%%%%%%%%%%%%%%%%%%%%%%%%%%%%%%%%%%%%%%%%%%%%%%%%%%%%%%%%%%%%%%%%%
% Principes
%%%%%%%%%%%%%%%%%%%%%%%%%%%%%%%%%%%%%%%%%%%%%%%%%%%%%%%%%%%%%%%%%%%%%%%%%%%%%%

\titreB {Principes}

\begin {frame} {Principes}

    Verrou d'exclusion mutuelle (\emph {mutex\/}) :
    mécanisme primitif de synchronisation

    \vspace* {3mm}

    Mécanisme de plus haut niveau~: 
    \emph {sémaphore} de Dijkstra

    \begin {itemize}
	\item un compteur
	\item 3 opérations atomiques
	    \begin {itemize}
		\item initialisation
		\item P
		\item V
	    \end {itemize}
    \end {itemize}

\end {frame}

\begin {frame} {Principes}

    \begin {itemize}
	\item Opération atomique P~:
	    \lstinputlisting [basicstyle=\scriptsize\lstmonstyle, firstline=1, lastline=5] {\inc/sem-act.c}

	    P = \emph {proberen} (tester) en néerlandais \\
	    Moyen mnémotechnique : P = «~Puis-je ?~»

	\item Opération atomique V~:
	    \lstinputlisting [basicstyle=\scriptsize\lstmonstyle, firstline=7, lastline=9] {\inc/sem-act.c}

	    V = \emph {verhogen} (incrémenter) en néerlandais \\
	    Moyen mnémotechnique : V = «~Vas-y !~»
    \end {itemize}

\end {frame}

\begin {frame} {Principes}
    Compteur = nombre d'exemplaires disponibles d'une ressource

    Exemple~:

    \begin {center}
	\includegraphics [width=.5\textwidth] {\inc/parking}
    \end {center}

    \begin {itemize}
	\item initialement, le compteur est initialisé à 12
	\item il reste 4 exemplaires disponibles de la ressource
	    «~place de parking~»
	\item barrière d'entrée = P \implique teste et décrémente le compteur
	\item barrière de sortie = V \implique incrémente le compteur

    \end {itemize}

\end {frame}

\begin {frame} {Principes}
    Deux cas principaux~:

    \begin {itemize}
	\item sémaphore général~: le compteur peut prendre n'importe
	    quelle valeur

	\item sémaphore binaire~: le compteur peut prendre la valeur
	    0 ou la valeur 1

	    \implique peut servir à réaliser une exclusion mutuelle

    \end {itemize}
\end {frame}

\begin {frame} {Exemple}

    Soit les deux processus A et B~:

    \begin {center}
	\begin {tabular} {c|c}
	    A & B \\ \hline
	    i$_1$ & i$_2$ \\
	\end {tabular}
    \end {center}

    On veut que i$_1$ soit exécutée avant i$_2$~:

    \begin {center}
	\begin {tabular} {c|c}
	    A & B \\ \hline
	    i$_1$ & P(s) \\
	    V(s) & i$_2$ \\
	\end {tabular}
    \end {center}

    Sémaphore s initialisé à 0
\end {frame}

%%%%%%%%%%%%%%%%%%%%%%%%%%%%%%%%%%%%%%%%%%%%%%%%%%%%%%%%%%%%%%%%%%%%%%%%%%%%%%
% Implémentation
%%%%%%%%%%%%%%%%%%%%%%%%%%%%%%%%%%%%%%%%%%%%%%%%%%%%%%%%%%%%%%%%%%%%%%%%%%%%%%

\titreB {Implémentation}

\begin {frame} {Implémentation naïve}

    Implémentation possible avec des \emph {spin mutexes}~:

    \lstinputlisting [basicstyle=\scriptsize\lstmonstyle] {\inc/sem-spin.c}

    \implique attente active + \emph {spin mutexes} (attente active également)

\end {frame}

\begin {frame} {Implémentation avec attente passive}

    Attente passive \implique support du système d'exploitation

    \vspace* {3mm}

    Modification du sémaphore~:
    \begin {itemize}
	\item compteur
	\item liste des processus en attente
    \end {itemize}
\end {frame}

\begin {frame} {Implémentation avec attente passive}

    Nouvelle définition de P et V~:

    \lstinputlisting [basicstyle=\scriptsize\lstmonstyle] {\inc/sem-passif.c}

    \begin {itemize}
	\item \code {compteur} > 0 \implique nombre de ressources disponibles
	\item \code {compteur} < 0 \implique nombre de processus en attente
	\item attention à l'atomicité de P et V
    \end {itemize}

\end {frame}

%%%%%%%%%%%%%%%%%%%%%%%%%%%%%%%%%%%%%%%%%%%%%%%%%%%%%%%%%%%%%%%%%%%%%%%%%%%%%%
% Semaphores IPC System V
%%%%%%%%%%%%%%%%%%%%%%%%%%%%%%%%%%%%%%%%%%%%%%%%%%%%%%%%%%%%%%%%%%%%%%%%%%%%%%

\titreB {Sémaphores IPC System V}

\begin {frame} {IPC System V}
    IPC System V : objet «~ensemble de sémaphores~»

    \vspace* {3mm}

    \begin {itemize}
	\item objet très sophistiqué
	\item opérations atomiques sur un ensemble de sémaphores

	    Exemple~: si je peux faire un P sur s[0] et sur s[1],
	    je fais également un V sur s[2]

	\item \implique interface de programmation complexe !
    \end {itemize}

\end {frame}

\begin {frame} {IPC System V}
    Primitives d'accès~:

    \begin {itemize}
	\item \code {semget}~: crée ou accède à un objet de type
	    «~ensemble de sémaphores~»
	\item \code {semctl}~: opérations (p. ex. initialisation ou
	    suppression) sur un ensemble de sémaphores

	    \vspace* {3mm}

	\item \code {semop}~: opérations P et V sur sémaphores
    \end {itemize}

\end {frame}


\begin {frame} [fragile] {IPC System V}
    Opérations spécifiques des sémaphores~:
    \begin {itemize}
	\item \code {semctl}~:
	    \begin {itemize}
		\item \code {GETVAL} / \code {SETVAL} : valeur (>0) d'un
		    sémaphore de l'ensemble
		\item \code {GETNCNT} : nombre de processus en attente
		\item \code {GETALL} / \code {SETALL}~: nécessite
		    un tableau pour les valeurs
	    \end {itemize}
	\item \code {semop}~: utilise un tableau de structures \code
	    {sembuf} \\
	    \implique spécifie les opérations à réaliser sur l'ensemble

\begin {lstlisting} [basicstyle=\scriptsize\lstmonstyle]
struct sembuf {
    unsigned short sem_num ;  /* numero de semaphore */
    short          sem_op ;   /* P:-1, V:+1 */
    short          sem_flg ;  /* 0 */
} ;
\end{lstlisting}

    \end {itemize}

\end{frame}

\begin {frame} {IPC System V}
    \lstinputlisting [basicstyle=\scriptsize\lstmonstyle, firstline=13, lastline=30] {\inc/ipc-sem.c}
\end {frame}

\begin {frame} {IPC System V}
    \lstinputlisting [basicstyle=\scriptsize\lstmonstyle, firstline=32, lastline=46] {\inc/ipc-sem.c}
\end {frame}

\begin {frame} {IPC System V}
    \lstinputlisting [basicstyle=\scriptsize\lstmonstyle, firstline=48, lastline=55] {\inc/ipc-sem.c}
\end {frame}

\begin {frame} {IPC System V}
    \lstinputlisting [basicstyle=\scriptsize\lstmonstyle, firstline=57, lastline=71] {\inc/ipc-sem.c}
\end {frame}


%%%%%%%%%%%%%%%%%%%%%%%%%%%%%%%%%%%%%%%%%%%%%%%%%%%%%%%%%%%%%%%%%%%%%%%%%%%%%%
% Semaphores IPC System V
%%%%%%%%%%%%%%%%%%%%%%%%%%%%%%%%%%%%%%%%%%%%%%%%%%%%%%%%%%%%%%%%%%%%%%%%%%%%%%

\titreB {Sémaphores POSIX}

\begin {frame} {Sémaphores POSIX}
    Sémaphores IPC System V : complexes pour les cas simples

    \vspace* {3mm}

    POSIX a normalisé une API simple et homogène \\
    plus proche de la définition de Dijkstra
\end {frame}

\begin {frame} {Sémaphores POSIX}
    Pour utiliser les sémaphores POSIX~:
    \begin {itemize}
	\item fichier d'inclusion : \code {\#include <semaphore.h>}
	\item édition de liens : \texttt {cc -o ...-l pthread}

	    Note~: l'API n'est pas spécifique aux threads \\
	    \implique peut être utilisée pour synchroniser des processus
    \end {itemize}

    Type~:

    \begin {itemize}
	\item \code {sem\_t}~: identificateur de sémaphore
    \end {itemize}

\end {frame}


\begin {frame} {Sémaphores POSIX}
    Fonctions de l'API~:

    \begin {itemize}
	\item création et initialisation de sémaphore~:

	    \code {\small int sem\_init (sem\_t *sem, int prtg, unsigned int val)} 

	    \vspace* {1mm}
	    \begin {itemize}
		\item \code {\small prtg} = 0 : accès limité aux threads du processus
		    courant
		\item \code {\small prtg} $\neq$ 0 : accès à tous
		    les processus qui partagent la mémoire \code {*sem}

	    \end {itemize}

	\item suppression de sémaphore~:

	    \code {\small int sem\_destroy (sem\_t *sem)}

	\item lecture du compteur~:

	    \code {\small int sem\_getvalue (sem\_t *sem, int *val)}

    \end {itemize}

\end {frame}

\begin {frame} {Sémaphores POSIX}
    Fonctions de l'API~:

    \begin {itemize}
	\item P~:

	    \code {\small int sem\_wait (sem\_t *sem)}

	\item P sans attente~:

	    \code {\small int sem\_trywait (sem\_t *sem)}

	\item P avec attente bornée~:

	    \code {\small int sem\_timedwait (sem\_t *sem, struct timespec *habs)}

	    Avec
	    \code {\small struct timespec \{ time\_t tv\_sec; long tv\_nsec; \}}
	    \\
	    \implique comme dans \code {\small pthread\_mutex\_timedlock}

	\item V~:

	    \code {\small int sem\_post (sem\_t *sem)}
    \end {itemize}

\end {frame}


%%%%%%%%%%%%%%%%%%%%%%%%%%%%%%%%%%%%%%%%%%%%%%%%%%%%%%%%%%%%%%%%%%%%%%%%%%%%%%
% Problèmes classiques de synchronisation
%%%%%%%%%%%%%%%%%%%%%%%%%%%%%%%%%%%%%%%%%%%%%%%%%%%%%%%%%%%%%%%%%%%%%%%%%%%%%%

\titreB {Problèmes classiques de synchronisation}

\begin {frame} {Problèmes classiques de synchronisation}

    Problèmes classiques de concurrence
    \\
    \implique Illustration des mécanismes de synchronisation

    \begin {enumerate}
	\item producteur/consommateur sur tampon infini (non borné)
	\item producteur/consommateur sur tampon borné
	\item lecteurs/écrivains avec priorité aux lecteurs
    \end {enumerate}

    \implique souvent rencontrés dans les cas réels

\end {frame}

\begin {frame} {Producteur/consommateur / tampon infini}

    Énoncé~:
    \begin {itemize}
	\item plusieurs processus produisent des données
	\item plusieurs processus consomment des données
	\item hypothèse irréaliste (mais pédagogique) : tableau de
	    taille infinie

    \end {itemize}

    \begin {center}
	\includegraphics [width=.7\textwidth] {\inc/prodcons-infini}
    \end {center}

\end {frame}

\begin {frame} {Producteur/consommateur / tampon infini}
    \lstinputlisting [basicstyle=\scriptsize\lstmonstyle] {\inc/prodcons-infini.c}
\end {frame}

\begin {frame} {Producteur/consommateur / tampon infini}
    Trois sémaphores, pour~:

    \begin {itemize}
	\item protéger les accès concurrents à l'indice
	    de la première case à consommer
	    \\
	    \implique sémaphore binaire

	\item protéger les accès concurrents à l'indice
	    de la prochaine case à produire
	    \\
	    \implique sémaphore binaire

	\item représenter le nombre de cases produites
	    (i.e. que l'on peut consommer)
	    \\
	    \implique sémaphore général

    \end {itemize}
\end {frame}

\begin {frame} {Producteur/consommateur / tampon borné}

    Énoncé~: idem précédemment, mais tampon de taille bornée

    \begin {center}
	\includegraphics [width=.7\textwidth] {\inc/prodcons-borne}
    \end {center}

\end {frame}

\begin {frame} {Producteur/consommateur / tampon borné}
    \lstinputlisting [basicstyle=\scriptsize\lstmonstyle] {\inc/prodcons-borne.c}
\end {frame}

\begin {frame} {Producteur/consommateur / tampon borné}
    Un sémaphore supplémentaire, pour~:

    \begin {itemize}
	\item représenter le nombre de cases libres
	    (i.e. que l'on peut produire)
	    \\
	    \implique sémaphore général

    \end {itemize}
\end {frame}

\begin {frame} {Lecteurs/écrivains avec priorité aux lecteurs}

    Énoncé~:

    \begin {itemize}
	\item plusieurs processus peuvent accéder à une donnée
	    \begin {itemize}
		\item les accès en lecture peuvent être concurrents
		\item un accès en écriture est exclusif de tout autre
		    accès \\
		    (en lecture comme en écriture)
	    \end {itemize}

	\item un écrivain ne peut accéder à la donnée que si plus
	    aucun lecteur n'y accède
	    \\
	    \implique priorité aux lecteurs

    \end {itemize}

\end {frame}

\begin {frame} {Lecteurs/écrivains avec priorité aux lecteurs}

    Problème typique de l'accès aux données complexes.

    \vspace* {3mm}

    Exemples~:
    \begin {itemize}
	\item \code {\small struct timespec}
	    \\
	    \implique si la donnée est modifiée entre la lecture de
	    \code {\small tv\_sec} et 
	    \code {\small tv\_nsec},
	    le résultat est inintelligible

	\item accès à une ligne d'une table de SGBD

    \end {itemize}

\end {frame}

\begin {frame} {Lecteurs/écrivains avec priorité aux lecteurs}
    \lstinputlisting [basicstyle=\scriptsize\lstmonstyle] {\inc/lect-ecr.c}
\end {frame}

\begin {frame} {Lecteurs/écrivains avec priorité aux lecteurs}
    Deux sémaphores, pour~:

    \begin {itemize}
	\item protéger les accès concurrents à \code {nlect}
	    \\
	    \implique sémaphore binaire

	\item protéger l'accès en écriture exclusivement de tout
	    autre accès
	    \\
	    \implique sémaphore binaire

    \end {itemize}
\end {frame}

