%
% Historique
%   2014/08/30 : pda : conception
%

\def\inc{inc5-interb}

\titreA {Interblocage}

%%%%%%%%%%%%%%%%%%%%%%%%%%%%%%%%%%%%%%%%%%%%%%%%%%%%%%%%%%%%%%%%%%%%%%%%%%%%%%
% Introduction
%%%%%%%%%%%%%%%%%%%%%%%%%%%%%%%%%%%%%%%%%%%%%%%%%%%%%%%%%%%%%%%%%%%%%%%%%%%%%%

\titreB {Introduction}

\begin {frame} {Introduction}
    Interblocage : situation dans laquelle un ou plusieurs processus
    sont bloqués et ne peuvent pas se débloquer eux-mêmes

    \vspace* {3mm}

    Autres noms~: \emph {deadlock}, étreinte mortelle, etc.

    \vspace* {3mm}

    \begin {itemize}
	\item comment caractériser un interblocage ?
	\item comment programmer sans interblocage ?
	\item quelles autres stratégies face à l'interblocage ?
    \end {itemize}

\end {frame}

\begin {frame} {Exemple d'interblocage}
    Exemple (avec des sémaphores binaires)~:

    \begin {center}
	\footnotesize
	\begin {tabular} {c|c}
	    Processus A & Processus B \\
	    \hline
	    P (r$_1$) & P (r$_2$) \\
	    P (r$_2$) & P (r$_1$) \\
	    V (r$_2$) & V (r$_1$) \\
	    V (r$_1$) & V (r$_2$) \\
	\end {tabular}
    \end {center}

    \begin {itemize}
	\item situation d'interblocage si A a obtenu la ressource r$_1$ et
	    B a obtenu la ressource r$_2$

	\item l'interblocage n'est pas obligatoire : si A obtient toutes
	    ses ressources, A finira bien par les libérer et B pourra être
	    débloqué
    \end {itemize}

\end {frame}

\begin {frame} {Exemple d'interblocage}
    Pas besoin de plusieurs processus pour avoir un interblocage
    \\
    \implique «~interblocage du pauvre~»

    \vspace* {3mm}

    Exemple (avec un sémaphore binaire)~:

    \begin {center}
	\footnotesize
	\begin {tabular} {c}
	    Processus A \\
	    \hline
	    P (r$_1$) \\
	    P (r$_1$) \\
	    V (r$_1$) \\
	    V (r$_1$) \\
	\end {tabular}
    \end {center}

    \implique avec un seul processus et une seule ressource, on peut
    arriver à une situation d'interblocage

\end {frame}

\begin {frame} {Exemple d'interblocage}
    Les sémaphores ne sont pas les uniques responsables des interblocages

    \vspace* {3mm}

    Exemple (avec une barrière initialisée à 2)~:

    \begin {center}
	\footnotesize
	\begin {tabular} {c|c}
	    Processus A & Processus B \\
	    \hline
	    mutex\_lock (\&v) & mutex\_lock (\&v) \\
	    barrier\_wait (\&b) & barrier\_wait (\&b) \\
	    mutex\_unlock (\&v) & mutex\_unlock (\&v) \\
	\end {tabular}
    \end {center}

    \implique interblocage certain car A (resp. B) acquiert le verrou
    v, attend ensuite que la barrière s'ouvre, alors que B (resp. A)
    attend le verrou v

    % \vspace* {3mm}
    %
    % Dans la suite, on ne parlera plus que de ressources

\end {frame}

%%%%%%%%%%%%%%%%%%%%%%%%%%%%%%%%%%%%%%%%%%%%%%%%%%%%%%%%%%%%%%%%%%%%%%%%%%%%%%
% Conditions de l'interblocage
%%%%%%%%%%%%%%%%%%%%%%%%%%%%%%%%%%%%%%%%%%%%%%%%%%%%%%%%%%%%%%%%%%%%%%%%%%%%%%

\titreB {Conditions de l'interblocage}

\begin {frame} {Conditions de l'interblocage}
    4 conditions sont nécessaires et suffisantes pour avoir une situation
    d'interblocage \\
    \implique conditions de Coffman, Elphick et Shoshani (1971)

    \begin {enumerate}
	\item exclusion mutuelle : une ressource ne peut pas être
	    allouée simultanément à plusieurs processus

	\item conserver et attendre~: lorsqu'un processus attend une
	    ressource, il ne libère pas celles qu'il a déjà obtenues

	\item pas de réquisition~: les ressources déjà acquises ne
	    peuvent pas être réquisitionnées

	\item attente circulaire~: P$_0$ attend une ressource obtenue par
	    P$_1$, qui attend lui-même une ressource obtenue par P$_2$...
	    par P$_n$ qui attend lui-même une ressource obtenue par P$_0$

    \end {enumerate}

\end {frame}

\begin {frame} {Conditions de l'interblocage}
    Retour sur le premier exemple~:
    \begin {center}
	\footnotesize
	\begin {tabular} {c|c}
	    Processus A & Processus B \\
	    \hline
	    P (r$_1$) & P (r$_2$) \\
	    P (r$_2$) & P (r$_1$) \\
	    V (r$_2$) & V (r$_1$) \\
	    V (r$_1$) & V (r$_2$) \\
	\end {tabular}
    \end {center}

    \begin {enumerate}
	\item exclusion mutuelle~: r$_1$ et r$_2$ ne sont
	    allouées qu'à un seul processus

	\item conserver et attendre~: A a déjà obtenu r$_1$ et la
	    conserve pendant l'attente de r$_2$ (idem pour B)

	\item pas de réquisition~: de par la conception des sémaphores

	\item attente circulaire~: A attend r$_2$, qui est obtenue
	    par B, qui lui-même attend r$_1$ qui est obtenue par A

    \end {enumerate}

    Les 4 conditions sont remplies \implique situation d'interblocage
\end {frame}

\begin {frame} {Conditions de l'interblocage}
    Retour sur le deuxième exemple~:
    \begin {center}
	\footnotesize
	\begin {tabular} {c}
	    Processus A \\
	    \hline
	    P (r$_1$) \\
	    P (r$_1$) \\
	    V (r$_1$) \\
	    V (r$_1$) \\
	\end {tabular}
    \end {center}

    \begin {enumerate}
	\item exclusion mutuelle~: r$_1$ n'est allouée qu'à A

	\item conserver et attendre~: A a déjà obtenu r$_1$ et la
	    conserve pendant l'attente de r$_1$

	\item pas de réquisition~: de par la conception des sémaphores

	\item attente circulaire~: A attend r$_1$, qui est obtenue
	    par A

    \end {enumerate}

    Les 4 conditions sont remplies \implique situation d'interblocage
\end {frame}

\begin {frame} {Graphe d'allocation de ressources}
    Description de l'interblocage avec un graphe d'allocation de
    ressources~:

    \begin {itemize}
	\item Graphe orienté $G = <S, A>$
	\item Deux types de sommets : $S = \{P, R\}$
	    \begin {itemize}
		\item $P = \{P_1, P_2, ... P_n \}$ : ensemble des processus
		\item $R = \{R_1, R_2, ... R_m \}$ : ensemble des ressources
		    \\
		    Chaque ressource possède un nombre d'exemplaires
	    \end {itemize}
	\item Deux types d'arcs $\in A$ :
	    \begin {itemize}
		\item Arc $R_j \rightarrow P_i$ : un exemplaire de la
		    ressource $R_j$ est allouée au processus $P_i$
		    \\
		    \implique l'arc part de l'exemplaire de la ressource

		\item Arc $P_i \rightarrow R_j$ : le processus $P_i$ attend
		    un exemplaire de la ressource $R_j$
		    \\
		    \implique l'arc arrive sur la ressource et non sur
		    un exemplaire spécifique
	    \end {itemize}
    \end {itemize}

\end {frame}

\begin {frame} {Graphe d'allocation de ressources}
    Exemple~:
    \begin {center}
	\includegraphics [width=.4\textwidth] {\inc/gralloc}
    \end {center}
    \begin {itemize}
	\item $P_1$ a acquis un exemplaire de $R_1$ et de $R_2$

	\item $P_2$ a acquis un exemplaire de $R_1$ et de $R_2$ et attend
	    un exemplaire de $R_3$

	\item $P_3$ a acquis un exemplaire de $R_3$ et attend
	    un exemplaire de $R_4$

    \end {itemize}
\end {frame}

\begin {frame} {Graphe d'allocation de ressources}
    Propriétés du graphe d'allocation de ressources~:

    \begin {itemize}
	\item interblocage \implique cycle dans le graphe
	\item dans le cas général (ressources à exemplaires
	    multiples), l'inverse n'est pas vrai : un cycle dans le graphe
	    n'implique pas forcément un interblocage

	\item dans le cas spécifique de ressources à exemplaire unique,
	    cycle dans le graphe \implique interblocage

    \end {itemize}
\end {frame}

\begin {frame} {Graphe d'allocation de ressources}
    Exemple de situation sans interblocage~:
    \begin {center}
	\includegraphics [width=.4\textwidth] {\inc/gr-0}
    \end {center}
    Dans cette situation~:
    \begin {itemize}
	\item il y a un cycle dans le graphe
	    ($P_1 \rightarrow R_2 \rightarrow P_2 \rightarrow R_1
	    \rightarrow P_1$)

	\item $P_1$ et $P_2$ sont effectivement bloqués
	\item $P_3$ n'est pas bloqué et peut se terminer, auquel
	    cas il libère la ressource $R_2$ qui permettra à $P_1$
	    de continuer
    \end {itemize}
    \implique pas d'interblocage
\end {frame}

\begin {frame} {Graphe d'allocation de ressources}
    Exemple avec des ressources à exemplaire unique~:
    \begin {center}
	\includegraphics [width=.4\textwidth] {\inc/gr-1}
    \end {center}
    \begin {itemize}
	\item les ressources sont à exemplaire unique
	\item il y a un cycle
	    ($P_1 \rightarrow R_3 \rightarrow P_3 \rightarrow R_2
	    \rightarrow P_2 \rightarrow R_1 \rightarrow P_1$)
	\item donc il y a interblocage
    \end {itemize}
\end {frame}

%%%%%%%%%%%%%%%%%%%%%%%%%%%%%%%%%%%%%%%%%%%%%%%%%%%%%%%%%%%%%%%%%%%%%%%%%%%%%%
% Graphe
%%%%%%%%%%%%%%%%%%%%%%%%%%%%%%%%%%%%%%%%%%%%%%%%%%%%%%%%%%%%%%%%%%%%%%%%%%%%%%

\titreB {Gestion de l'interblocage}

\begin {frame} {Gestion de l'interblocage}
    Comment gérer un interblocage ?

    \vspace* {3mm}

    Plusieurs approches possibles~:
    \begin {enumerate}
	\item ignorer l'interblocage
	\item prévenir l'interblocage
	\item éviter que l'interblocage ne se produise
	\item détecter l'interblocage et intervenir
    \end {enumerate}
\end {frame}

\begin {frame} {Ignorer l'interblocage}
    Solution la plus simple

    \begin {itemize}
	\item adoptée par la plupart des systèmes d'exploitation \\
	    Exemples : Unix, Windows

	\item c'est à l'application de gérer l'interblocage
    \end {itemize}

    \implique les autres approches sont beaucoup plus complexes

\end {frame}

\begin {frame} {Prévenir l'interblocage}
    Prévenir l'interblocage : faire en sorte qu'une des 4 conditions
    soit toujours non vérifiée

    \begin {enumerate}
	\item exclusion mutuelle : offrir des ressources sans accès
	    exclusif \\
	    Exemple~: spooler d'imprimante, fichier en lecture seule

	\item conserver et attendre~: relâcher toutes les ressources
	    déjà obtenues en cas d'attente \\
	    Exemple~: noyau Unix, allouer toutes les ressources en
	    une seule fois de manière atomique (sémaphores System V)

	\item pas de réquisition~: permettre la réquisition \\
	    Exemple~: pas d'exemple

	\item attente circulaire~: ordonner les allocations \\
	    Exemple~: programmer les allocations toujours dans le
	    même ordre

    \end {enumerate}

\end {frame}

\begin {frame} {Éviter l'interblocage}
    Éviter l'interblocage : éviter de se placer dans une situation
    où l'interblocage pourrait peut-être survenir dans le futur

    \begin {minipage} {.25\textwidth}
	\includegraphics [width=\textwidth] {\inc/gr-sur}
    \end {minipage}
    \begin {minipage} {.73\textwidth}
	\vspace* {2mm}
	Si $P_1$ et $P_2$ demandent \emph {encore} chacun~:
	\begin {itemize}
	    \item 1 exemplaire de $R_1$ :
		\\
		\implique interblocage évité si $P_1$ (resp. $P_2$)
		    termine et libère ses deux exemplaires
		\\
		\implique l'état actuel est «~sûr~»

	    \item 2 exemplaires de $R_1$ :
		\\
		\implique interblocage certain \\
		\implique l'état actuel est «~non sûr~»

	\end {itemize}
    \end {minipage}

    \vspace* {3mm}


    Si on connaît à l'avance toutes les demandes futures des processus,
    on peut éviter les états (états non sûrs) dans lesquels un
    interblocage ultérieur est possible
    \\
    \implique chaque processus doit déclarer toutes les ressources
    nécessaires à son exécution

\end {frame}

\begin {frame} {Éviter l'interblocage}
    Notion d'état sûr~:

    \begin {quote}
	Un état est sûr si et seulement si il existe une séquence de
	processus $P_1, P_2, ..., P_n$ telle que $\forall i$, la demande
	de $P_i$ puisse être satisfaite par les ressources déjà
	obtenues par $P_i$ ainsi que les ressources libérées par les
	$P_j$ avec $j < i$

    \end {quote}

    \begin {center}
	\includegraphics [width=.7\textwidth] {\inc/etat-sur}
    \end {center}

\end {frame}

\begin {frame} {Éviter l'interblocage}
    Comment éviter l'interblocage ? \\
    \implique vérifier, lors de l'allocation, que l'état résultant
    reste sûr

    \begin {enumerate}
	\item en utilisant le graphe d'allocation des ressources \\
	    (pour les ressources à un seul exemplaire)
	\item en utilisant l'algorithme du banquier \\
	    (pour le cas général)
    \end {enumerate}

    \vspace* {3mm}

    Ces méthodes nécessitent de connaître à priori toutes les
    ressources nécessaires pour chaque processus

\end {frame}

\begin {frame} {Éviter l'interblocage -- Graphe d'allocation}
    Principe~:

    \begin {itemize}
	\item matérialiser les demandes futures de ressources...
	\item ... via un arc spécial $P_i \leadsto R_j$
	    \\
	    (figuré graphiquement par une flèche pointillée)
	\item quand $R_j$ est allouée à $P_i$ ou si $P_i$ attend $R_j$,
	    l'arc spécial est remplacé par un arc classique
	\item l'état reste sûr si et seulement si il n'y a pas de
	    cycle dans le graphe résultant
	\item si ce n'est pas le cas, l'allocation est bloquante
	\item lorsque $P_i$ libère $R_j$, l'arc $R_j \rightarrow P_i$
	    est remplacé par l'arc spécial original $P_i \leadsto R_j$

    \end {itemize}
\end {frame}

\begin {frame} {Éviter l'interblocage -- Graphe d'allocation}
    Exemple~: $P_1$ et $P_2$ ont chacun déclaré l'utilisation de
    $R_1$ et $R_2$

    \begin {center}
	\includegraphics [width=.8\textwidth] {\inc/evit-graphe}
    \end {center}

    \begin {enumerate}
	\item initialement, arcs $P_i \leadsto R_j$ : demandes futures

	\item $P_1$ demande ensuite $R_1$ : il l'obtient (arc $R_1
	    \rightarrow P_1$)

	\item après, $P_2$ demande $R_2$~: l'arc $R_2 \rightarrow P_2$
	    créerait un cycle \\
	    \implique demande bloquée : arc $P_2 \leadsto R_2$
		remplacé par $P_2 \rightarrow R_2$

	\item $P_1$ alloue ensuite $R_2$, termine son exécution et
	    libère $R_1$ et $R_2$. Les arcs initiaux $P_1 \leadsto R_1$ 
	    et $P_1 \leadsto R_2$ sont rétablis \\
	    \implique $P_2$ peut reprendre (et obtenir $R_2$)

    \end {enumerate}
\end {frame}

\begin {frame} {Éviter l'interblocage -- Algor. du banquier}
    Avec des ressources à exemplaires multiples, un cycle dans le
    graphe d'allocation de ressources ne permet pas de conclure à
    un interblocage ou à un état non sûr

    \vspace* {3mm}

    Principe de l'algorithme du banquier~:

    \begin {enumerate}
	\item simuler l'allocation de la ressource demandée
	\item trouver un processus qui peut obtenir toutes ses
	    allocations
	\item s'il en existe un, alors simuler son exécution jusqu'à
	    la fin : simuler la libération de ses ressources déjà
	    obtenues et recommencer en 2
	\item s'il ne reste plus aucun processus, l'état est sûr
	\item s'il reste des processus qui ne peuvent obtenir leurs
	    ressources, l'état n'est pas sûr
    \end {enumerate}

    \implique comme un vrai banquier, l'algorithme n'alloue que ce qu'il
    pourra récupérer

\end {frame}

\begin {frame} {Éviter l'interblocage -- Algor. du banquier}
    Notations~:
    \begin {itemize}
	\item $P_1 ... P_i ... P_n$ : processus ($i$ : index d'un $P_i$)
	\item $R_1 ... R_j ... R_m$ : ressources ($j$ : index d'un $R_j$)
	\item $k_j$ : nombre d'exemplaires de $R_j$
    \end {itemize}

    \vspace* {3mm}

    Signification et valeur initiale des variables~:

    \begin {itemize}
	\item dispo [$j$] = $k_j$ : nombre d'exemplaires disponibles de $R_j$
	\item max [$i, j$] : demande maximum de $P_i$ pour $R_j$
	\item alloc [$i, j$] = 0 : nombre d'exemplaires de $R_j$ alloués à $P_i$
	\item encore [$i,j$] = max [$i, j$] : demande de $P_i$ pour $R_j$ pour
	    permettre à $P_i$ de se terminer
	    \\
	    Propriété : encore [$i,j$] = max [$i, j$] - alloc [$i, j$] 
    \end {itemize}
\end {frame}

\begin {frame} {Éviter l'interblocage -- Algor. du banquier}
    Allocation (req [] = vecteur de demande d'allocation de $P_i$)~:

    \footnotesize
    \begin {tabbing}
	\hspace* {3mm} \= \hspace* {3mm} \= \hspace* {3mm} \=
		\hspace* {40mm} \= \kill
	\> si $\neg$ (req [$j$] $\leq$ encore [$j$], $\forall j$) \\
	    \> \> alors erreur \\
	\> fin si \\
	debut: \\
	\> si $\neg$ (req [$j$] $\leq$ dispo [$j$], $\forall j$) \\
	\> alors attendre \\
	\> sinon \\
	\> \> dispo [$j$] -= req [$j$], $\forall j$
		\> \> /* simuler une allocation */ \\
	\> \> alloc [$i, j$] += req [$j$], $\forall j$ \\
	\> \> encore [$i, j$] -= req [$j$], $\forall j$
		\> \> /* fin de la simulation */ \\
	\> \> si l'état simulé est sûr
		\> \> /* test état sûr, cf page suivante */ \\
	\> \> alors allouer les ressources \\
	\> \> sinon \\
	\> \> \> défaire la simulation \\
	\> \> \> attendre \\
	\> \> \> goto debut \\
	\> \> fin si \\
	\> fin si \\
    \end {tabbing}
\end {frame}

\begin {frame} {Éviter l'interblocage -- Algor. du banquier}
    Test de sûreté~:

    \footnotesize
    \begin {tabbing}
	\hspace* {3mm} \= \hspace* {3mm} \= \hspace* {3mm} \=
		\hspace* {40mm} \= \kill
	fini [$i$] = faux, $\forall i$
	    \> \> \> \> /* simulation : tous les $P_i$ sont actifs */
	    \\
	tmp [$j$] = dispo [$j$], $\forall j$ \\
	tant qu'il existe $i$ tel que
	    (fini[$i$] = faux $\wedge$
		encore [$i,$j] $\leq$ tmp [$j$] $\forall j$) \\
	faire \\
	\> tmp [$j$] = tmp [$j$] + alloc [$i,j$], $\forall j$
	    \> \> \> /* simuler la libération des ressources de $i$ */
	    \\
	\> fini [$i$] = vrai
	    \> \> \> /* simuler la terminaison de $P_i$ */
	    \\
	fin tant que \\
	si fini [$i$] = vrai $\forall i$ \\
	alors l'état est sûr \\
	sinon l'état n'est pas sûr
	    \> \> \> \> /* il reste des $P_i$ qui n'ont pu se terminer */
	    \\
	fin si
    \end {tabbing}
\end {frame}


\begin {frame} {Détecter l'interblocage}
    Solution «~à priori~» simple~:

    \begin {enumerate}
	\item laisser l'interblocage arriver
	\item détecter l'interblocage
	    \begin {itemize}
		\item quand ? lors de l'allocation ? examen périodique ?
		\item comment ? détection coûteuse
	    \end {itemize}
	\item réagir suite à l'interblocage
	    \begin {itemize}
		\item réquisitionner des ressources ? il faut que
		    l'application soit prévue pour cela

		\item terminer des processus ? lesquels choisir ? tous
		    ? seulement quelques-uns afin de casser l'interblocage ?

	    \end {itemize}
    \end {enumerate}

    \implique solution pas si simple...

\end {frame}
